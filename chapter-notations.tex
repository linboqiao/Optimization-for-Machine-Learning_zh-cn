\chapter*{
    % Notation
    符号说明
    }

% We use the following mathematical notation in this writeup:
本书中使用的数学表示列举如下:
\begin{itemize}
\item
% $d$-dimensional  Euclidean space is denoted $\reals^d$. 
$d$ 维 欧几里得空间表示为 $\reals^d$。
\item
% Vectors are denoted by boldface lower-case letters such as $\x \in \reals^d$.  Coordinates of vectors are denoted by underscore notation $\x_i$ or regular brackets $\x(i)$.
向量 表示为加粗小写字母,如 $\x \in \reals^d$。向量的下标 表示为下划线符号 $\x_i$ 或常规的括号 $\x(i)$。
\item
% Matrices are denoted by boldface upper-case letters such as $\mathbf{X}  \in \reals^{m \times n}$.  Their coordinates by $\mathbf{X}(i,j)$, or $\mathbb{X}_{ij}$. 
矩阵 表示为加粗大写字母,如 $\mathbf{X}  \in \reals^{m \times n}$。矩阵的下标表示为 $\mathbf{X}(i,j)$,或者 $\mathbb{X}_{ij}$。
\item
% Functions are denoted by lower case letters $f: \reals^d \mapsto \reals$. 
函数 表示为小写字母 $f: \reals^d \mapsto \reals$。
\item 
% The $k$-th differential of function $f$ is denoted by $\nabla^k f \in \reals^{d^k}$.  The gradient is denoted without the superscript, as $\nabla f$. 
函数的$k$-阶倒数 表示为 $\nabla^k f \in \reals^{d^k}$。函数的梯度 则表示为 $\nabla f$ (没有上标)。
\item
% We use the mathcal macro for sets, such as $\K \subseteq \reals^d$.  
集合用 mathcal宏 来表示,如 $\K \subseteq \reals^d$。
\item
% We denote the gradient at point $\x_t$ as $\nabla_{\x_t}$, or simply $\nabla_t$. 
在点$\x_t$处的梯度表示为 $\nabla_{\x_t}$,或简化为 $\nabla_t$。
\item
% We denote the global or local optima of functions by $\x^\star$. 
函数的全局或者局部最优解 表示为 $\x^\star$。
\item
% We denote distance to optimality for iterative algorithms by $h_t = f(\x_t) - f(\x^\star)$. 
迭代算法中与最优函数值的距离 表示为 $h_t = f(\x_t) - f(\x^\star)$。
\item
% Euclidean distance to optimality is denoted $d_t = \|\x_t - \x^\star\| $.
最优解的欧几里得距离表示为 $d_t = \|\x_t - \x^\star\| $。
\item 
oracle model, 先知模型
\end{itemize}
