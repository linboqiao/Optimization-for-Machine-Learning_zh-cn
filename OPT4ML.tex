\documentclass[11pt]{book}

%\usepackage[margin=1in, paperwidth=6in, paperheight=9in]{geometry}

\usepackage{amsmath}
\usepackage{amsthm}
\usepackage{amsfonts}
%\usepackage{latexsym}
\usepackage{amssymb}
\usepackage{url}
\usepackage{hyperref}
\usepackage{algorithm,algorithmic}
\usepackage{xspace}
\usepackage{color}
\usepackage{mathtools}
\usepackage{enumerate}
%\usepackage[numbers,comma,sort&compress]{natbib} 

\newtheorem{theorem}{Theorem}[chapter]
\newtheorem{definition}[theorem]{Definition}
\newtheorem{example}[theorem]{Example}
\newtheorem{proposition}[theorem]{Proposition}
\newtheorem{corollary}[theorem]{Corollary}
\newtheorem{lemma}[theorem]{Lemma}
\newtheorem{remark}[theorem]{Remark}
\newtheorem{application}[theorem]{Application}
\newtheorem{optimizationproblem}[theorem]{Optimization~Problem}
\newtheorem{fact}[theorem]{Fact}
\newtheorem{assumption}[theorem]{Assumption}

\usepackage[utf8]{inputenc}
\usepackage{CJKutf8}


\widowpenalty10000
\clubpenalty10000

% the now document class sets various dimensions, so be sure to *not* set
% or alter dimensions in your latex code.
% be sure to remove all manual formatting commands such \newpage, \clearpage.

% a few definitions that are *not* needed in general:
\input{defs}

\title{ 
%{\it lecture notes: }   \\  Optimization for Machine Learning \\ 
机器学习最优化讲义 \\ 

{\small \it version 0.57}\\  \bigskip
{\small All rights reserved.  }
}

\date{}
\author{
作者:Elad Hazan \thanks{\url{www.cs.princeton.edu/\~ehazan}}\\
Princeton University, 
ehazan@cs.princeton.edu\\ \\

翻译:乔林波\\
国防科技大学,qiao.linbo@nudt.edu.cn
}

\begin{document}
\begin{CJK*}{UTF8}{gbsn}


\frontmatter  % title page, contents, catalog information

\maketitle



\input{"preface.tex"}

\tableofcontents

\mainmatter


\input{"chapter-notations.tex"}


%%%%%%%%%%%%%%%%%%%%%%%%%%%%%%%%%%%%%%%%%%%%%%%%%%%%%%%%%%%%
%%%%%%%%%%%%%%%%%%%%%%%%%%%%%%%%%%%%%%%%%%%%%%%%%%%%%%%%%%%%
%  Introduction
%%%%%%%%%%%%%%%%%%%%%%%%%%%%%%%%%%%%%%%%%%%%%%%%%%%%%%%%%%%%
%%%%%%%%%%%%%%%%%%%%%%%%%%%%%%%%%%%%%%%%%%%%%%%%%%%%%%%%%%%%
\input{"chapter01.tex"}
\input{"chapter02.tex"}
\input{"chapter03.tex"}
\input{"chapter04.tex"}
%\input{"chapter5.tex"}
%\input{"chapter6.tex"}
%\input{"chapter7.tex"}
%\input{"chapter8.tex"}
%\input{"chapter9.tex"}
%\input{"chapter10.tex"}
%\input{"chapter11.tex"}

\backmatter
\bibliographystyle{plain}
\bibliography{bookbib}

%\appendix
%\input{appendix}

\end{CJK*}
\end{document}
